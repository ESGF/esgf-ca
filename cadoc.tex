\textbf{Certificate Policy and Certificate Practices Statement Template for ESGF CAs}

(In RFC 3647 format)

\textbf{Date:} \today\\
\textbf{Version:} \versionnum

\section{Introduction}\label{introduction}

The Earth System Grid Federation (ESGF), uses three internal Certification Authorities (CAs), to issue CA and identity certificates for the systems in ESGF.
These CAs are operated by
\begin{enumerate}
\item Argonne National Laboratory (ANL), Lemont, IL, USA
\item Institut Pierre Simon Laplace (IPSL), Paris, France.
\item National Supercomputer Centre (NSC), Link\"oping, Sweden.
\end{enumerate}
This document describes the set of rules and procedures established by ESGF, for the operation of its CAs. Structured according to RFC 3647 (\url{https://www.ietf.org/rfc/rfc3647.txt}), this document describes policy and practices to be followed by ESGF CAs. Each individual ESGF Root CA is expected to have its own Certificate Policy and Certificate Practices Statement (CP/CPS). It may derive from this document.
The CA/CPS is designed so that a Relying Party can look at them and obtain an understanding of the trustworthiness of credentials issued by the ESGF CAs. The most current version of this document will be available at \url{https://github.com/ESGF/esgf-ca}

\subsection{Overview}\label{overview}

The ESGF CA infrastructure supports grid and e-science activities in
the Earth System Grid Federation (ESGF). This document describes the
set of rules and procedures established by ESGF for the operations of its CAs. The purpose of the ESGF CAs is:

\begin{enumerate}
\item
  
  To sign host certificates for nodes operated by vetted members of the ESGF federation.
  
\item
  
  To sign subordinate CA certificates for nodes operated by vetted members of the federation, with the understanding that the subordinate CA may only be used to issue end-user credentials and short-lived credential proxies. 
  
\end{enumerate}

\subsection{Document name and identification}\label{document-name-and-identification}

This document is the CP and CPS Template of the ESGF CAs.

\textbf{Document title:} Certificate Policy and Certification Practice Statement Template for ESGF CAs\\
\textbf{Document version:} \versionnum\\

\textbf{Document date:} \today

\subsection{PKI participants}\label{pki-participants}

\begin{enumerate}
\item
  
  Certificate Authorities: ANL, IPSL, and NSC issue CA certificates and host certificates for the ESGF.
  
\item
  
  Registration Authorities: There are no RAs external to the issuing
  authorities. The individual CAs are responsible for all approvals and revocations that they perform.
  
\item
  
  Subscribers: Only institutions that participate in the ESGF receive
  certificates from the Root CA.
  
\item
  
  Relying Parties: No stipulation.
  
\end{enumerate}

\subsection{Certificate usage}\label{certificate-usage}

The ESGF Root CA certificate may be used for the following purposes:

\begin{enumerate}
\item
  
  To validate the signature of a Subject CA.
  
\item
  
  More generally, to validate any certificate chain ending with this
  Root, provided all certificates in the chain are used for permitted
  purposes.
  
\item
  
  To validate the signature on a CRL issued by this Root. No other use
  of the ESGF Root CA certificate is permitted. The ESGF Root CA
  asserts that all Subordinate CAs serve the same community and they all
  issue in distinct namespaces.
  
\end{enumerate}

\subsection{Policy administration}\label{policy-administration}

The Argonne National Laboratory (ANL), and National Supercomputer Center (NSC) at Link\"oping University (LiU) are responsible for drafting,
registering, maintaining and updating this CP/CPS template. The persons
responsible for this policy and the practices of the CA are: 
\begin{enumerate}
\item Lukasz Lacinski, Argonne National Laboratory (ANL), Building 240, 9700 Cass Avenue, Lemont, IL 60439, Email esgf-ca@lists.llnl.gov.
\item Prashanth Dwarakanath, National Supercomputer Center (NSC), House Galaxen, Entrance 83, Link\"oping University, Campus Valla, Olaus Magnus v\"ag 42-44, Link\"oping, Email: esg-admin@nsc.liu.se.
\end{enumerate}

\subsection{Definitions and acronyms}\label{definitions-and-acronyms}

\begin{enumerate}
\item ANL: Argonne National Laboratory, located in Lemont, IL, USA.
\item IPSL: Institut Pierre Simon Laplace, located in Paris, France.
\item NSC: National Supercomputer Center at Link\"oping University (LiU), located in Link\"oping, Sweden.

\item EE: Users obtaining credentials from a Subordinate/Subject CA are called End Entities (EE). 
  
\item
  
  ESGF: is the Earth System Grid Federation.

\item RP:  Relying party (RP) is a computer term used to refer to a server providing access to a secure software application. Claims-based applications, where a claim is a statement an entity makes about itself in order to establish access, are also called relying party applications.
  
\item
  
  For the purposes of this document, a Subject CA is a CA whose
  certificate was issued by the Root whose policy and practices are
  described in this document.
  
\item
  
  For the purposes of this document, a Subordinate CA is a Subject CA in
  a hierarchy whose root is described in this document.
 
\item
  
  For the purposes of this document, Profile refers to the content of
  the signed envelope within a certificate, but excluding the public key
  itself and the lifetime. Thus, Profile normally comprises extensions,
  the issuer and subject names, but also the type of keys and
  algorithms, and the version of the certificate.
  
\end{enumerate}

\section{Publication and repository responsibilities}\label{publication-and-repository-responsibilities}

It is the responsibility of each of the ESGF Root CAs to publish the
following information on their respective websites

\begin{enumerate}
\item
  
  Its CP/CPS;
  
\item
  
  Its certificate;
  
\item
  
  All certificates issued by the ESGF Root CA and their status;
  
\item
  
  The signing policy of the hierarchy of which it forms the root;
  
\item
  
  Its Certificate Revocation List (CRL)
  
\end{enumerate}

The ESGF Root CA grants the ESGF the right of unlimited
redistribution of this information.

\section{Identification and authentication}\label{identification-and-authentication}

\subsection{Naming}\label{naming}

\begin{enumerate}
\item
  
  Each of the Subject CAs shall have a unique name;
  
\item
  
  Its Subject DN shall form the Subject name of each Subject CA to
  relate its purpose and distinguish it from others alone.
  
\item
  
  The Issuer name shall include /O=ESGF.
  
\item
  
  No subject name of a Subject CA shall be reused anywhere in the
  hierarchy.
  
\item
  
  Subject CA names have a fixed and a variable component. The
  certificate subject names start with the fixed component to which a
  variable component is appended to make it unique. The fixed component
  is as follows:
  
\end{enumerate}

/O=ESGF/OU=ESGF.ORG

\subsection{Initial identity validation}\label{initial-identity-validation}

A certificate shall be issued to a Subject CA only when:

\begin{enumerate}
\item
  
  The requestor has been ascertained to be the responsible node administrator of the node the certificate is being sought for.
  
\item
 The requestor and his organization have been vetted to be a part of the ESGF federation.
 
\item  
  The Subject CA promises to adhere to the policies and practices described in this document.
  
\item
  
  The Subject CA has submitted a certificate request and is able to
  prove to the Root CA possession of the corresponding private key.
  
\end{enumerate}

Furthermore, the ESGF Root CA requires, as a condition for
certificate issuance, that:

\begin{enumerate}
\item
  
  All Subject CAs make available to the ESGF Root CA results of CA
  audits and plans to remedy deficiencies
  
\item
  
  All Subject CA Managers and Operators agree to be signed up to a
  closed mailing list, if such a mailing list is maintained by the ESGF Root CA;
  
\item
  
  The Subject CA's certificate request (and hence certificate) contains
  no personal information.
  
\end{enumerate}

\subsection{Identification and authentication for re-key requests}\label{identification-and-authentication-for-re-key-requests}

The CA manager of the subject CA shall prove
possession of the private key corresponding to the certificate being
renewed, and prove possession of the private key corresponding to the
request being submitted.

\subsection{Identification and authentication for revocation request}\label{identification-and-authentication-for-revocation-request}

The certificate of a Subject CA will be revoked when:

\begin{enumerate}
\item
  
  A revocation request is received which is signed with the private key
  of the Subject CA; or,
  
\item
  
  An authenticated revocation request from the CA Manager of the Subject
  CA is received; or,
  
\item
  
  The ESGF Root CA has otherwise determined the need for revocation,
  e.g., if the Subject CA does not comply with the requirements imposed
  on it by the ESGF Root CA.
  
\end{enumerate}

\section{Certificate life-cycle operational requirements}\label{certificate-life-cycle-operational-requirements}

The ESGF CA should use a key of at least 4096 bits have a lifetime of at least five years. For Subject CAs, keys are generated by the requestors who are responsible for the security of the keys themselves. The Subject CAs shall have a lifetime not exceeding three years.

\subsection{Certificate application}\label{certificate-application}

For an initial request, the Manager of the Subject CA shall agree to the
namespace of the Subject CA with the Manager of the ESGF Root CA. It is
the responsibility of the Manager of the Subject CA to ensure that it
operates within the constraints imposed by the CP/CPS of the ESGF Root CA.

\subsection{Certificate application processing}\label{certificate-application-processing}

When the ESGF Root CA Manager is satisfied that Subject CA will
operate within the constraints imposed by the Root, The ESGF Root CA
will issue and publish the certificate of the Subject CA.

\subsection{Certificate issuance}\label{certificate-issuance}

The ESGF Root CA Manager makes the Subject CA available on its
website.

\subsection{Certificate acceptance}\label{certificate-acceptance}

Both the ESGF Root CA Manager and the ESGF CA Subject CA Manager
shall verify the content of the Subject CA certificate against the
CP/CPS of the ESGF Root CA and test its use (e.g. against grid
middleware). If problems are discovered during testing, the certificate
shall be revoked by the ESGF Root CA, and reissued with changes,
provided that these changes are still compatible with the CP/CPSes of
ESGF Root CA.

\subsection{Keypair and certificate usage}\label{key-pair-and-certificate-usage}

The certificates of all Subordinate CAs and those of the EEs issued by
Subordinate CAs must be used only for purposes of direct academic
science and grid work, or related incidental support (i.e.
infrastructure, email). The certificates issued to Subject CAs may only
be used as CA certificates, i.e., for validating certificates issued by
them, and for validating CRLs. A Subordinate CA may impose further
constraints on the use of certificates on, and only on, their EEs.
Conversely, no Subordinate CA shall relax constraints imposed on its
policy or operations by the CP/CPS of a CA of which it is itself
Subordinate. It is the responsibility of the EE to use certificates for
permitted purposes only. It is the responsibility of RPs to validate the
certificate to their satisfaction at the time of reliance.

\subsection{Certificate renewal}\label{certificate-renewal}

Issued certificates are never automatically renewed, and it is the responsibility of the certificate requestors to contact the ESGF CAs by emailing to esgf-ca@lists.llnl.gov and requesting for a new certificate. Requestors are expected to generate new keys and make a fresh request for a new certificate.

\subsection{Certificate re-key}
No stipulation.

\subsection{Certificate modification}
Certificate modificatiions are neither offered nor supported. A new certificate has to be requested and issued, and the old one revoked.

\subsection{Certificate revocation and suspension}\label{certificate-revocation-and-suspension}

A Subject key CA shall be revoked if:

\begin{enumerate}
\item
  
  It is seen to violate requirements imposed on it by the policy and
  practices of the ESGF Root CA; or
  
\item
  
  It can be shown that the private key has been compromised.
  
\end{enumerate}

\subsection{Certificate status services}\label{certificate-status-services}

The Root CA shall issue a CRL. Certificates and certificate status of
Subject CAs are available on the Root CA's web site. See also section
\ref{ca-or-ra-termination}.

\subsection{End of subscription}\label{end-of-subscription}

No stipulation.

\subsection{Key escrow and recovery}

The ESGF Root CA key shall be encrypted using at leaast 16-character alphanumeric random passphrase, and backed up on a CD and stored in a tamper-evident envelope securely stored with the organization's ESGF admin officer.

\section{Facility, management and operational controls}\label{facility-management-and-operational-controls}

This section discusses specific ESGF CA procedures related to its
facility and ESGF Root CA operation.

\subsection{Physical security}\label{physical-security-controls}

The machine on which the Root CA signs its certificates and CRLs is
expected to be installed in a facility which restricts physical access of the machine only to designated personnel.

\subsection{Procedural controls}
No stipulation.

\subsection{Personnel controls}\label{personnel-controls}

Training: The Root CA is OpenSSL based, and ESGF CA Managers must
have sufficient experience with OpenSSL to be able to issue certificates
and CRLs. ESGF CA Managers must be permanent staff of the respective institution.

\subsection{Audit logging procedures}\label{audit-logging-procedures}

All OpenSSL operations and basic system logs for the signing machine
will be saved in files that are periodically signed and backed up onto secondary storage media.

\subsection{Records archival}\label{records-archival}

Records are kept throughout the lifetime of the CA and for a period no
less than three years after the termination of the CA.

\subsection{Key changeover}\label{key-changeover}

At re-keying, the new Root public keys shall be published on the Root
CA's web site, as certificates signed by both the old and the new
private key. The transitional certificate, signed with the old key,
shall expire at the same time as the old Root certificate, but shall
otherwise have the same content as the new Root certificate. It shall be
clearly marked as a transitional certificate, and instructions shall be
provided for users explaining how to verify the transition.

\subsection{Compromise and disaster-recovery}\label{compromise-and-disaster-recovery}

Following any compromise of the Root private key, The Root CA shall make
this widely known to all peer CAs, Subject CAs and relying parties.
Efforts to re-issue new Subject CA certificates will follow the method
described in section \ref{ca-or-ra-termination}.

\subsection{CA or RA termination}\label{ca-or-ra-termination}

Upon termination of the Root CA, a ESGF CA Manager shall communicate
this in advance to peer CAs, Subject CAs and relying parties. The
advance notice should be no less than the longest lifetime of any
currently valid Subject CA.

\section{Technical security controls}\label{technical-security-controls}

This section discusses technical aspects specific to the operation of
the ESGF Root CA.

\subsection{Key-pair generation and installation}\label{key-pair-generation-and-installation}

The Root CA's key pair shall be generated with sufficient entropy, by the CA. It shall be the responsibility of a ESGF CA Manager to generate the key pair. The Root key pair shall be RSA and have a length of at least 4096 bits. Key pairs for Subject CAs are
generated according to best practices. Each Subject CA key pair should
have a length of at least 2048 bits.

\subsection{Private key protection and cryptographic module engineering controls}\label{private-key-protection-and-crypto}
The ESGF Root CA Manager is responsible for the security of the private key of the Root CA, and any copies, including any held in escrow. Recommendations to secure the Root CA server on the network are found in section \ref{network-security-controls}.

\subsection{Other aspects of key pair management}\label{other-aspects-of-key-pair-management}

All Root CA certificates shall be kept and published throughout the
lifetime of the CA, and a period no less than three years after the
termination of the CA. Subject CAs' key pairs shall have a lifetime not
to exceed one year.

\subsection{Activation data}\label{activation-data}
No stipulation.

\subsection{Computer security controls}\label{computer-security-controls}

Only authorized personnel may access the dedicated ESGF CA server.

\subsection{Lifecycle technical controls}\label{life-cycle-technical-controls}

Only authorized personnel may perform hardware maintenance or upgrade software on the dedicated ESGF CA server.

\subsection{Network security controls}\label{network-security-controls}
\begin{enumerate}
\item ESGF Root CA server should be on a private LAN with no other hosts on the same network, except the gateway host which it may use in order to download critical updates.
\item It's strongly advised that the gateway host only allow packet forwarding for the CA server, when updates are to be installed, ensuring that it is totally isolated from the internet at all other times. 
\item A local firewall policy on the server that prohibits all machines other than the gateway host machine, to connect to it via ssh, is also recommended. 
\item No other ports on the server should be open from any machine, including the gateway host machine. This ensures that the ESGF Root CA server is disconnected from the internet and all other machines, except the host machine, at all times, other than the time when it is being updated.
\item Even when the ESGF Root CA server is being updated, only outbound connection requests from the server ought to be  permitted to receive responses, i.e. it cannot be contacted by any external machine, directly, at any time.
\end{enumerate}

\subsection{Time-stamping}\label{time-stamping}

No timestamping service is offered, but ESGF Root CA managers are responsible to ensure that the ESGF Root CA server maintains correct time, using any means necessary, including manual adjustment. This is to ensure that the generated credentials have proper `valid from' and `valid until' values.
The CA server is expected to maintain correct time using NTP service, to stay in sync with designated NTP servers. The selection of the NTP servers is managed by each individual CA, based on their network policies. The ESGF Root CA Manager shall periodically check the time on the server, to ensure that it is really in sync, and correct it, only if necessary.

\section{Certificate, CRL and OCSP profiles}\label{certificate-crl-and-ocsp-profiles}

This section articulates details of certificates and certificate
revocation lists issued by the ESGF Root CA. The ESGF Root CA
does not currently provide OCSP support.

\subsection{Certificate profile}\label{certificate-profile}

All certificates issued by the ESGF Root CA conform to the Internet
PKI profile (PKIX) for X.509 certificates as defined by RFC 3280
{[}RFC3280{]}.

The Root CA shall have the following Profile:

\begin{enumerate}
\item
  
  The certificate shall be version 3 (i.e., the version number shall be
  2);
  
\item
  
  The issuer and subject name shall both contain the following:
O = ESGF, OU = ESGF.ORG 
  
\item
  
  The signature algorithm shall be sha256WithRSAEncryption;
  
\item
  
  The extensions shall contain:
  

  \begin{enumerate}
  \item
    
    basicConstraints: CA=true, critical;
    
  \item
    
    keyUsage: certificate signing, CRL signing, critical;
    
  \item
    
    There shall be a subjectKeyIdentifier and it shall have the hash as a value;
    
  \end{enumerate}
\end{enumerate}

The requirements on the Profile of the Subject CAs are as follows:

\begin{enumerate}
\item
  
  The certificate shall be version 3 (i.e., the version number shall be
  2);
  
\item
  
  basicConstraints must be present and critical and must contain
  CA=true, ( but may contain other constraints);
  
\item
  
  keyUsage must be present and critical and must have certificate
  signing and CRL signing set, and no other value;
  
\item
  
  There shall be a subjectKeyIdentifier containing the hash.
  
\item
  
  Other extensions are allowed.
  
\end{enumerate}

\subsection{CRL Profile}\label{crl-profile}

The Root CA issues CRL version 2 (i.e., the version number shall be 1).
The ``lifetime'' of the CRL is 18 months and it is issued at least once
every year.

\subsection{OCSP Profile}\label{ocsp-profile}

Not applicable.

\section{Compliance audit and other assesments}\label{compliance-audit-and-other-assessments}

The organization's security contact/ESGF Admin shall carry out a compliance audit, as directed by the ESGF Software Security Working Team (SSWT), and shall report the findings back to SSWT.

\section{Other business and legal matters}\label{other-business-and-legal-matters}

The section headers in section \ref{other-business-and-legal-matters} are taken from RFC3647 and are kept
as-is for ease of reference and comparison with other CAs. They must not
be interpreted or construed in any way that will affect the
interpretation or construction of the contents of the sections.
Certificates and all other components of the CA must be used for lawful
purposes only. CA Managers shall sign a document to the effect that they
will comply with the procedures and requirements described in this
document.

\subsection{Fees}\label{fees}

The ESGF Root CA charges no fees for its services.

\subsection{Financial responsibility}\label{financial-responsibility}

No financial responsibility is accepted for certificates issued under
this policy.

\subsection{Confidentiality of business information}\label{confidentiality-of-business-information}

The ESGF Root CA will follow best practices to protect any
confidential information as well as policies specified by the individual institutions.

\subsection{Privacy of personal information}\label{privacy-of-personal-information}

The ESGF Root CA does not process any personal data, except for the
following:

\begin{enumerate}
\item
  
  The email addresses of the ESGF site administrators who request certificates from ESGF Root CAs. These are not published and are used only for announcements pertaining
  to the Root CA or announcements affecting all Subject CAs.
  
\end{enumerate}

ESGF Root CA will publish a generic email address to contact
ESGF Root CA Officers: esgf-ca@lists.llnl.gov.

\subsection{Intellectual property rights}\label{intellectual-property-rights}

This document is based on RFC3647 and its application to the ``UK
e-Science Root Certificate Policy and Certification Practices
Statement'' by the UK e-Science CA run by CCLRC. The ESGF Root CA does not claim any IPR on certificates that it has issued. Anybody may
freely copy from any version of the ESGF Root CA's Certificate
Policy and Certification Practices Statement provided they include an
acknowledgment of these sources.

\subsection{Representations and warranties}\label{representations-and-warranties}

When issuing a certificate to a Subject CA, the ESGF Root CA provide a current copy of the ESGF Root CA CP/CPS, which the requestor is expected to read and comply with.

\subsection{Disclaimers of warranties}\label{disclaimers-of-warranties}

ESGF CA makes no representation and gives no warranty, condition or
undertaking in relation to the ESGF Root CA and its operation.

\subsection{Limitations of liability}\label{limitations-of-liability}

With respect of the information published by the Root CA, including, but
not limited to certificates and CRLs, the Root CA shall make best
endeavors to ensure the information is timely and accurate. ESGF CA
shall be under no obligation or liability, and no warranty condition or
representation of any kind is made, given or to be implied as to the
sufficiency, accuracy or fitness for purpose of such information. The
recipient party, whether CA, RP, EE, or anyone else, shall in any case
be entirely responsible for the use to which it puts such information.
The ESGF Root CA also declines any liability for damages arising
from the nonissuance of a requested certificate, or for the revocation
of a certificate initiated by the CA or the designated RA acting in
conformance with this CP/CPS.

\subsection{Indemnities}\label{indemnities}

The ESGF Root CA declines any payment of indemnities for damages
arising from the use or rejection of certificates it issues. Each
Subject CA and relying party shall indemnify and hold harmless ESGF CA
and keep ESGF CA indemnified against any and all damages, costs, claims
or expenses, which are awarded against, or suffered by the Subject CA or
relying parties or their hosting institution or company, as a result of
any act or omission of the relying party or Subject CA.

\subsection{Term and termination}\label{term-and-termination}

The initial lifetime of the ESGF CA is five years, and it is intended to be renewed for a similar period, prior to reaching expiration. Should the ESGF Root CA service be permanently discontinued, there shall be a federation-wide announcement, prior to closure.

The ESGF Root CA shall issue no certificate whose lifetime will exceed the date of termination and is obligated to maintain its CRL until its termination.

\subsection{Individual notices and communications with participants}\label{individual-notices-and-communications-with-participants}

All agreements between the ESGF Root CA and any organization must be
documented and signed by the appropriate authorities. A mailing list
shall be maintained for announcements pertaining to the ESGF Root CA, or announcements affecting all Subject CAs.

\subsection{Amendments}\label{amendments}

The ESGF Root CA shall communicate amendments to Subject CAs and to
the ESGF.

\subsection{Dispute resolution provisions}\label{dispute-resolution-provisions}

ESGF Root CA managers shall resolve any disputes arising out of the
CP/CPS.

\subsection{Governing law}\label{governing-law}
The ESGF Root CA must comply with the business practices of the organization as well as the laws of the country it operates within.

\subsection{Compliance with applicable law}\label{compliance-with-applicable-law}

All activities relating to the request, issuance, use or acceptance of an ESGF Root CA certificate must comply with the laws of the country it operates within.

\subsection{Miscellaneous provisions}\label{miscellaneous-provisions}

No stipulation.

\subsection{Other provisions}\label{other-provisions}

No stipulation.

\section{Change log}\label{change-log}

\newpage
\section{ESGF Certificate Policy and Certificate Practices Statement: Review and Approval}
This Certificate Policy and Certificate Practices Statement for ESGF CAs for the Earth System Grid Federation, was prepared for the exclusive use of ESGF and in particular, its sites and developers. I have reviewed and concur with the contents of this plan.
\par\vspace{1cm}
\begin{tiny}

\textbf{Approved by/Date:} \underline{\hspace{4cm}}\today\underline{\hspace{4cm}}
\par\hspace{3.5cm}{\ttfamily Dean Williams, DOE
\par\hspace{3.5cm}ESGF Chair}

\textbf{Approved by/Date:} \underline{\hspace{4cm}}\today\underline{\hspace{4cm}}
\par\hspace{3.5cm}{\ttfamily Michael Lautenschlager, DKRZ
\par\hspace{3.5cm}ESGF Co-Chair}

\textbf{Approved by/Date:} \underline{\hspace{4cm}}\today\underline{\hspace{4cm}}
\par\hspace{3.5cm}{\ttfamily Sebastien Denvil, IPSL
\par\hspace{3.5cm}ESGF Executive Committee Member}

\textbf{Approved by/Date:} \underline{\hspace{4cm}}\today\underline{\hspace{4cm}}
\par\hspace{3.5cm}{\ttfamily Martin Juckes, STFC
\par\hspace{3.5cm}ESGF Executive Committee Member}

\textbf{Approved by/Date:} \underline{\hspace{4cm}}\today\underline{\hspace{4cm}}
\par\hspace{3.5cm}{\ttfamily Luca Cinquini, NASA/NOAA
\par\hspace{3.5cm}ESGF Executive Committee Member}

\textbf{Approved by/Date:} \underline{\hspace{4cm}}\today\underline{\hspace{4cm}}
\par\hspace{3.5cm}{\ttfamily Robert Ferraro, NASA
\par\hspace{3.5cm}ESGF Executive Committee Member}

\textbf{Approved by/Date:} \underline{\hspace{4cm}}\today\underline{\hspace{4cm}}
\par\hspace{3.5cm}{\ttfamily Daniel Q. Duffy, NASA
\par\hspace{3.5cm}ESGF Executive Committee Member}

\textbf{Approved by/Date:} \underline{\hspace{4cm}}\today\underline{\hspace{4cm}}
\par\hspace{3.5cm}{\ttfamily Cecilia DeLuca, NOAA
\par\hspace{3.5cm}ESGF Executive Committee Member}

\textbf{Approved by/Date:} \underline{\hspace{4cm}}\today\underline{\hspace{4cm}}
\par\hspace{3.5cm}{\ttfamily V. Balaji, NOAA
\par\hspace{3.5cm}ESGF Executive Committee Member}

\textbf{Approved by/Date:} \underline{\hspace{4cm}}\today\underline{\hspace{4cm}}
\par\hspace{3.5cm}{\ttfamily Ben Evans, NCI
\par\hspace{3.5cm}ESGF Executive Committee Member}

\textbf{Approved by/Date:} \underline{\hspace{4cm}}\today\underline{\hspace{4cm}}
\par\hspace{3.5cm}{\ttfamily Clair Trenham, NCI
\par\hspace{3.5cm}ESGF Executive Committee Member}

\end{tiny}
